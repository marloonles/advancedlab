\documentclass{article}
\usepackage{mathtools}
\usepackage{graphicx}
\usepackage[margin=1in]{geometry}
\usepackage{tabularx}
\usepackage{hyperref}
\hypersetup{
    colorlinks=true,
    linkcolor=blue
    filecolor=magenta}
\graphicspath{ {./downloads/} }
\title{Lab 3: Make a Heater and Measure Temperature}
\author{Marlene McKinney}

\begin{document}
\maketitle
\section{Introduction and Methods}
Current going through a resistor has the ability to generate enough heat to be measured. With a potentiometer, one can adjust the current that goes through the resistor. A thermistor then can be used to record the temperature changes resulting from the current changes. These elements attached to the breadboard are shown in the circuit diagram in Figure 1 and in the Arduino setup in Figure 2.

\begin{figure}[h!t]
\includegraphics[width=6cm]{Advanced Lab 1 Lab 3 Circuit}
\centering
\caption{Circuit diagram of the potentiometer, resistor, and thermistor arrangement.}
\end{figure}

\begin{figure}[h!t]
\includegraphics[width=6cm]{Adv Lab 1 Lab 3 Arduino}
\centering
\caption{Arduino and breadboard setup for thermistor measurements.}
\end{figure}

\section{Results and Discussion}
The resulting curve from the data collection can be found in Figure 3. Since the Arduino code started the collection at 0 for the potentiometer, it's likely that the plots being inverse of one another was because I started turning the potentiometer in the counterclockwise direction rather than clockwise.

\begin{figure}[h!t]
\includegraphics[width=11cm]{Potentiometer Graph}
\centering
\caption{Graph of thermistor and potentiometer value as a value of time.}
\end{figure}

\section{Plotting Code}
The data and code for the plot in the Results section along with the LaTeX code for this document can be found on \href{https://github.com/marloonles/advancedlab/tree/main}{GitHub}.
\end{document}